\documentclass[a4paper]{article}

\usepackage[top=2.5cm,bottom=2.5cm]{geometry}
\usepackage{graphicx}
\usepackage{hyperref}

%\bibliographystyle{naturemagurl}
\bibliographystyle{plainnat}


\title{Notes on histogram reweighting}
\author{Jacob Stevenson}
\date{\today}

\begin{document}
\maketitle


\section{parallel tempering}

In this case the windows (labelled $r$) each have temperature $T_r$.
A histogram of observed data is recorded where $N_{ir}$ is the number of times
replica $r$ was in energy bin $i$ and $N_r = \sum_i N_{ir}$ and $N = \sum_{ir}  N_{ir}$.
We know that the probability of being in energy bin $i$ given a temperature is
the product of the density of states $g_i$ and the Boltzmann weight.
\begin{equation}
p(i | r) = g_i e^{- E_i / T_r} / Z_{r}
\end{equation}
where $Z_{r}$ is the unknown normalization constant.

We ultimately want to find the values of $g_i$ and $Z_{r}$ that make $p(i|r)$ as 
close as possible to $N_{ri} / N$.  This can be done by making the ratio of the probabilities
as close to one as possible, or equivalently by minimizing the squared deviation of the logarithms for each bin.
\begin{equation}
R_{ri}^2 = \left[ \log(g_i) - \log(Z_{r}) - E_i/T_r - \log(N_{ir} / N_r) \right]^2
\end{equation}
We combine these together weighted by $N_{ri}$ so that the bins with the most observations contribute the most to the 
estimation of $g_i$ and $Z_r$.
\begin{equation}
\chi^2 = \sum_r \sum_i N_{ir} R_{ij}^2 = \sum_{ir} N_{ir} \left( \log \frac{g_{i} e^{- E_i/T_r}}{Z_r} - \log \frac{N_{ir}}{N_r} \right)^2
\end{equation}
We find the $g_i$ and $Z_r$ that minimize $\chi$.

\section{umbrella sampling}

In this situation we do simulations (indexed with $r$) each weighted by a constraint function $U_r(Q)$ where
$Q$ is some order parameter.  The constraint, for example, could be a squared exponential decay 
$U_r(Q) = e^{ -\alpha (Q - \hat{Q}_r )^2}$.  
Again, we record a histogram $N_{iqr}$ where the order parameter is indexed by $q$ and
the energy is indexed by $i$.
Again, we can decompose the joint probability of begin in energy bin $i$ and order parameter bin $q$ into
\begin{equation}
p(i,q|r) = g_{iq} e^{-E_i / T_r} U_r(Q_q) / Z_{r}
\end{equation}
where $g_{iq}$ is the unweighted density of states with energy in bin $i$ and order parameter in bin $q$.
As in above, we want $p(i,q|r)$ to be as close as possible to $N_{iqr}/N_r$, so we minimize
\begin{equation}
\chi^2 = \sum_{iqr} N_{iqr} \left( \log \frac{g_{iq} e^{- E_i/T_r} U_r(Q_q) }{Z_r} - \log \frac{N_{iqr}}{N_r} \right)^2
\end{equation}

The unweighted probability as a function of the order parameter evaluated at temperature $T$ is then obtained by summing over the energy bins
\begin{equation}
p_q \propto \sum_i g_{iq} e^{- E_i/T}
\end{equation}
$T$ does not have to be one of the replica temperatures $T_r$, but for accurate results it shouldn't be far from the temperatures studies. 
The free energy is simply $- k_B T \log(p_q)$.  

\section{constant temperature umbrella sampling}

If the simulations are all done at the same temperature $T_r = T$, then we can sum out the dependence on $i$
analytically
\begin{equation}
p(q|r) = \sum_i p(i,q|r) = p_q U_r(Q_q) / Z_{r}
\end{equation}
In this case the histogram does not need to include dependence on the energy, and the minimization becomes.
\begin{equation}
\chi^2 = \sum_{qr} N_{qr} \left( \log \frac{p_{q} U_r(Q_q) }{Z_r} - \log \frac{N_{qr}}{N_r} \right)^2
\end{equation}

\section{biasing the energy and order parameter}

\subsection{the wrong way}

In this case our bias function $U_r(Q_q, E_i)$ is dependent on the bin $i$ as well as the bin $q$.
\begin{equation}
p(i,q|r) = g_{iq} e^{-E_i / T} U_r(E_i, Q_q) / Z_{r}
\end{equation}
Now, even though all the temperatures are the same, summing out the 
dependence on $i$ becomes more complicated.
\begin{equation}
p(q|r) = \tilde{g_{q}} / Z_{r}
\end{equation}
where
\begin{equation}
\tilde{g_q} = \sum_i g_{iq} e^{-E_i / T} U_r(E_i, Q_q) \propto \sum_i p(i,q) U_r(E_i, Q_q) = p(q) \sum_i p(i|q) U_r(E_i, Q_q)
\end{equation}
Maybe if we sum out $i$ we don't need to record the histogram as a function of $i$, but we must
instead record $\sum_i p(i|q) U_r(E_i, Q_q)$ which is the \emph{unbiased} average of the bias function in each $q$ bin.
During the trajectory we draw sample configurations $X_j$ from the \emph{biased} probability distribution.
In order to do un-biased average we need to weight each configuration by $U_r^{-1}(E(X_j), Q(X_j))$
\begin{equation}
\sum_i p(i|q) U_r(E_i, Q_q) = \sum_{j\in q} \frac{U_r(E(X_j), Q(X_j))}{U_r(E(X_j), Q(X_j))} = N_{qr}
\end{equation}
But this doesn't really help us.

\subsection{the right way}

Instead of recording the histogram of visits, we instead record the histogram of
\begin{equation}
\hat{p}_{qr} = \frac{\sum_{j\in q} U_r^{-1}(E(X_j), Q(X_j))}{\sum_{j} U_r^{-1}(E(X_j), Q(X_j))}
\end{equation}
Since the configurations $X_j$ are drawn from the biased distribution with binned probability density function $p(i,q) U_r(E(X_j), Q(X_j)$,
the inverse bias term above sample average cancels and $\hat{p}_{qr}$ can be interpreted as the \emph{unbiased} probability of being in bin $q$ in simulation $r$.

So, we record $\hat{p}_{qr}$ during the trajectory and compare it to
\begin{equation}
p(q|r) = p_{q} / Z_{r}
\end{equation}
As before, we find $p_q$ and $Z_r$ which minimizes
\begin{equation}
\chi^2 = \sum_{qr} p_{qr} \left( \log \frac{p_{q}}{Z_r} - \log p_{qr} \right)^2
\end{equation}


\section{two order parameters with bias on one of them}

In this case we have two order parameters $S(x)$ and $Q(x)$ which, when binned,
are indexed by $s$ and $q$.  We simulate using a bias function $U_r(E, S)$.
The simple counts refer to the probability distribution
\begin{equation}
p(i,s,q|r) = g_{isq} e^{-E_i / T} U_r(E_i, S_s) / Z_{r}
\end{equation}
What we want to determine is
\begin{equation}
p_q = \sum_{is} p(i,s,q) \propto \sum_{is} g_{isq} e^{-E_i / T}
\end{equation}
Computing this via population average during a trajectory we have
\begin{equation}
\hat{p}_{qr} = \frac{\sum_{j\in q} U_r^{-1}(E(X_j), S(X_j))}{\sum_{j} U_r^{-1}(E(X_j), S(X_j))}
\end{equation}
which, if accumulated during the trajectories can be used to solve for $p_q$ as described above.

\subsection{if the bias is a hard constraint}

If the bias is a hard constraint, e.g. $U_r(E(X), S(X)) = \Theta(S(X)-S_{r}^{min}) \Theta(S_{r}^{max}-S(x))$, the above might not work.
This is because the trajectories are not ergodic, and some configurations necessary for the accurate calculation of $\hat{p}_{qr}$
will never be sampled.

\section{maximum likelihood formulation}

Here we use the parallel tempering example where for each trajectory we draw configurations
from the probability distribution
\begin{equation}
p_r(x) = e^{- E(x) / T_r} / Z_{r}
\end{equation}
with 
\begin{equation}
Z_r = \int p_r(x) dx
\end{equation}
In this formulation we are not doing any binning.  $g(E(x))$ is begin left as a function, but in
order to use all the information from all the trajectories we need to know the normalization constants
$Z_r$.  So we will use maximum likelihood to solve for those.
If a configuration is observed in trajectory $r_1$ then we can use it in trajectory $r_2$ if we re-weight 
the observation by $p_{r_2}(x) / p_{r_1}(x)$

The full likelihood is then
\begin{equation}
L = \prod_{r_1} \prod_{j=1}^{N_{r_1}} \prod_{r_2} p_{r_2} (x_j) / p_{r_1} (x_j) 
= \prod_{r_1} \prod_{j=1}^{N_{r_1}} \prod_{r_2} \frac{q_{r_2}(x_j) Z_{r_2}^{-1}}{q_{r_1}(x_j) Z_{r_1}^{-1}} 
\end{equation}
If we associate each configuration $x_j$ with the simulation at which it was sampled $q_j$, then the above can
be simplified
\begin{equation}
L = \prod_{j=1}^{N_{x}} \prod_{r} \frac{q_{r}(x_j) Z_{r}^{-1}}{q_{j}(x_j) Z_{j}^{-1}} 
\end{equation}

\subsection{this section is wrong}

\begin{equation}
L = \prod_{j=1}^{N_x} \prod_{r} g(E(x_j)) e^{-E(x_j) / T_{j} } / Z_{r}
\end{equation}
where $N_{x} = \sum_r N_r$ is the total number of configurations. 
\begin{equation}
\log L =
\sum_{r} \sum_{j=1}^{N_{x}} ( \log g(E(x_j)) -E(x_j) / T_{j} - \log Z_{r} )
\end{equation}
\begin{equation}
\log L =
N_R \sum_{j=1}^{N_{x}} ( \log g(E(x_j)) -E(x_j) / T_{j})
- N_{x} \sum_{r} \log Z_{r}
\end{equation}
where $N_R$ is the number of replicas (number of different simulations) and 


\end{document}
